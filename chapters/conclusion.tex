In this thesis the studies of various aspects of nuclear spallation
have been undertaken. Among broad spectrum of not well understood
phenomena related to this kind of reaction the four problems were
addressed.

\begin{enumerate}
\item The proceeding of the initial phase of the proton target nucleus collision.
It is commonly accepted hypothesis, that energy dissipation inside
the struck nucleus and observed abundant emission of fast particles
in spallation reaction is an effect of the intranuclear cascade of
binary interactions among the target nucleus constituents.

\vspace{0.3cm}

Nucleons and pions are the main products of the intranuclear
cascade, i.e. of the initial phase of the reaction. Thus, reliable
experimental data for these particles, measured in broad range of
their energies  are necessary to study the initial phase of the
reaction. New experimental data ($d^2\sigma/d\Omega dE$) for
production of $H$ isotopes and the charged pions in the $p$ + $Nb$
reaction at 3.5 GeV proton bombarding energy were obtained. These
data were measured in angular range from 20$^{\circ}$ to
80$^{\circ}$ of the laboratory emission angle $\theta$. The high
acceptance and magnetic field of HADES spectrometer permitted to
obtain the cross-section distributions exceeding the energy ranges
of the data available up to now in the literature. It was achieved
for almost all detected particles and detection angles. The most
significant extension of the measured kinematic region was obtained
for proton data.
%
The quality of the achieved distributions has been verified by their
comparison to the other world data of the similar type available in
the scientific literature.

The experimental data have been compared to the results of
calculations performed with the use of three contemporary
theoretical models (GiBUU, UrQMD, INCL++) commonly used in nuclear
and particle physics. All these models adopt the assumption of the
intranuclear cascade as a sequence of binary collisions among the
reaction participants embedded in the target nucleus. The INCL++
model is additionally equipped with the hypothetical mechanizm of
surface coalescence permitting the dynamical formation of light
nuclear clusters.

The qualitative as well as the quantitative analysis of the
predictive power of theoretical models has been performed. Despite
that all the tested models are able to reproduce the shapes of the
experimental excitation functions the magnitudes of the theoretical
cross-sections differ from the data in almost all examined cases by
the factors up to $\sim$ 2.

It can be concluded that the examined theoretical models do not
contain all physics ingredients demanded for description of the
experimental spectra of $\pi$, $p$, $d$ and $t$ with precision
better in average than factor $\sim$2.

\item The emission of nuclear fragments of very broad spectrum of masses. It is believed that this is an effect of
the deexcitation of the after-cascade remnant nucleus and the
responsible mechanism acts in the second step of spallation
reaction.

\vspace{0.3cm}

% The residual nuclei of the spallation reactions are
% mainly products of the second stage of the proton - nucleus
% collisions because the excited remnant of the intranuclear cascade
% still is able to evaporate particles, undergo (multi-)fragmentation
% or at some conditions may fission.
The total production cross
sections of heavy and intermediate mass products of the reaction
contain therefore information on both, first and second stage of the
spallation reaction.  In the present work the predictive power of
four theoretical models of the second stage of the spallation
reaction (ABLA07, GEM2, GEMINI++, SMM) have been studied. For this
purpose their ability to reproduce the total reaction cross-sections
for the production of isotopes of the atomic numbers of 3 $\leq$ $Z$
$\leq$ 56 from  $^{136}Xe$ + $p$ collisions at an energy of 1
GeV/nucleon \cite{napolitani2007measurement} was investigated. In all cases the
first step of the reaction has been simulated with the use of INCL++
model.

Significant differences between the results of the models and
discrepancies with experimental data have been found. %In average
%they are of factor $<$ 2.
They were quantified with the use of the
$A$-factor - a tool allowing the numerical comparison of the
predictive power of theoretical models.

It was realized that for the calculation of cross-sections for
isotope production in the very broad range of masses the predictive
power of GEMINI++ is the highest whereas the GEM2 results differ
most significantly from the data and from the results of other
models.

\item The yields of the particles resulting from the first and second phase of the reaction. The contribution of the
non-equlibrium and equilibrium emission to the total cross-section
for the given isotope
%or given elemental nucleus
is a complicated function of the isospin of emitted particles.

\vspace{0.3cm}

Qualitative properties of the spectra and angular
distributions of intermediate mass fragments (particles heavier than
$^{4}$He but lighter than fission fragments)  indicate in
 many spallation reactions the contribution
of a non-equilibrium mechanism to the reaction. Such a phenomenon
has been studied  for the angular distributions published in \cite{Green1984}
where production of Li, Be, B, C, N, O, F, Ne, Na and Mg isotopes
was investigated in $p$ + $Ag$ collisions at a proton beam energy of
480 MeV.


The equilibrium and non-equilibrium components of the cross-section
were established by integration of angular distributions of
 INCL++ and GEMINI++ supplied by phenomenological model of moving source with
 parameters fitted to experimental angular distributions,
respectively. The remarkable dependence of the relative yield of
both emission classes on the mass $A$ of the emitted fragments and
their third component of isospin \( T3 = (N - Z) / A \) was
observed.

\item The variation of the total production cross-section known as an Odd-Even Staggering.
This effect is dependent on the charge and mass relations of the
emitted reaction products. Most likely this phenomenon is relevant
to the available density of states during de-excitation of the
excited after-cascade remnant nucleus.

\vspace{0.3cm}

The dependence of the variation of the total cross-section for
fragment emission in the spallation reaction on the third component
of isospin $T3$, called the Odd-Even Staggering (OES) has been
realized and examined for the data obtained from \cite{Green1984}.

The examination of the ability of three theoretical models of
nucleus de-excitation (ABLA07, GEMINI++, SMM) coupled to INCL++, to
reproduce the observed staggering effect has been done. The $\delta$
function proposed in \cite{Mei_OES} was calculated both for experimental
cross-sections as well as for theoretical ones. For fragments of $N$
= $Z$ the shape of all model $\delta$-function is almost similar to
that of the experimental one, however the disagreement in magnitudes of
all of them has been observed. For emitted particles of $N$ $\neq$
$Z$ the deviation between the experimental and theoretical
$\delta$-functions is even stronger. It indicates that all examined
models neglect the phenomena responsible for experimentally observed
OES of isotopic cross-sections.

\end{enumerate}



At the current stage of the theoretical examination of the nuclear
spallation reactions induced by protons it seems that the precision
of the theoretical models providing solution for different classes
of involved processes is still not sufficient in order to identify
the exact mechanisms responsible for observed experimental effects
and to define the range of their applicability and contribution to
the production cross-section.

Apparently the physics ingredients in all used  in this thesis
theoretical models are sufficient to describe the shapes of the
experimental spectra of spallation products. However their
magnitudes can be predicted only within the precision of factor
usually about 2.

Provided here new experimental cross-sections of high quality and
performed critical analysis of the predictive power of various
theoretical models of nuclear spallation can be an important
contribution to the further extension of  understanding of nuclear
spallation physics.


