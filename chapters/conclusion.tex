\section{Validation of total cross-section}
	The following analyzes were performed using the p+Ag and p+ 136 Xe
	data:
	Investigation of the quality of reproduction of the isotopic production cross sections for emission of different isotopes of the Li, Be, ... Ba nuclei from collisions of 1 GeV protons with 136 Xe by combination of two models: The INCL++ model describing fast, non-equilibrium phase of the collisions coupled to ABLA07,GEM,GEMINI and SMM models which were used to reproduce the second phase of the process, i.e. -emission of products from equilibrated remnant of the first stage of the processes.(This is the content of published paper \cite{singh2018predictive})
	\section{Validation of differential cross-section for IMF and heavy products}
	Analysis of angular distributions for double differential cross sections $d\sigma/d\Omega dE$ for intermediate mass fragments (isotopes of Li, Be, ... Mg elements) emitted from p+Ag reaction at E p =0.48 GeV with the aim to extract:
	\begin{itemize}
		\item Contribution of the non-equilibrium processes to the reaction, 
		\item Dependence of the isotopic total cross sections of these reactions on the izospin degree of freedom.
	\end{itemize}
	
	(This is the content of [UNPUB] paper prepared for publication in EPJA but
	rejected by referees)
	Investigation of the odd-even staggering in the yields of IMFs from the above reac-
	tion, i.e. p+Ag collisions at E p =0.48 GeV with the aim to study a possibility to
	validate models of the spallation reactions using this degree of freedom. (This is
	the content of Acta Phy)
	\section{LCP data analysis and validation of different models}
	Determination of differential cross sections $d\sigma/d\Omega dE$ for emission of Hydrogen isotopes as well as pions ( $\pi^+$ and $\pi^-$ ) using raw data stored event-by-event from collisions of protons with Nb nuclei at E p =3.5 GeV in HADES collaboration experiment performed at GSI in Sep 2008. Theoretical analysis of these data by means of following theoretical models like INCL++ ,BUU, UrQMD, JAM.
	The above investigations of HADES data were performed with the aim to shed light on the first, non-equilibrium part of the proton-nucleus collisions whereas the analysis of the literature data for p+Ag and p+ 136 Xe gave a chance to obtain information on
	the mechanism of the second, slow stage of the collision process which appears after equilibration of the excited remnant of the target nucleus.