\section{Spallation reaction}

The nuclear spallation is a sequence of reactions 
induced when a target nucleus is struck by an incident particle of energy greater than around 50 MeV. In such a collision the emission of nucleons, light charged particles (LCP), i.e., particles heavier than proton but lighter than $^5$He, and heavier nuclei may be expected besides elastic and inelastic collision of the projectile with the target nucleus. The nucleus resulting from the first collision is usually excited and may emit further ejectiles during its rearrangement. Therefore various nuclear processes may be involved, which usually lead to a significant modification of the initial nucleus.

%The defnition of nuclear spallation given by Nuclear Physics Academic Press \cite{} is as follows: \\
%\textit{ "A type of nuclear reaction in which the high-energy of incident particles causes the nucleus to eject more than three particle, thus changing both its mass number and its atomic number."}
\subsection{Why study nuclear spallation?}

Understanding these processes is interesting, but there are also practical reasons to undertake the studies of nuclear spallation. The gained knowledge can be used in  
numerous scientific, technological and medical applications. 
Only a few of them are mentioned below. They are:
%The spallation reaction plays very important role in a wide domain of applications which are like 
\begin{itemize}
    \item construction of the neutron sources used for scientific studies and technology applications  \cite{carpenter1977pulsed};
    \item conversion of long-living nuclear waste into nonradioactive substances or short life products \cite{ravn98A};
    \item production of short-living rare isotopes for medical science studies \cite{bowman1992nuclear} and scientific purposes \cite{meneguzzi1971production};
    \item construction of various kinds of radiation protection to be used on Earth or in the spacecraft;
    \item simulation and construction of detection apparatus for nuclear and particle physics.
\end{itemize}

%The simulation tools developed for these domains apply nuclear model codes to compute the production yields and characteristics of all the particles and nuclei generated in these reactions.\par
%The recent activities in production of Rare Isotope Beams and Spallation Sources led to revival of interest in reliable and predictive simulation of collisions of hadron-nucleus and nucleus-nucleus in the energy range of few hundred MeV to few GeV per particle, to be embedded in transport codes (e.g. MCNPX, GEANT). 

%Owing to the complexity of the quantum-mechanical many-body problems, the processes are often solved approximately.

\subsection{The need for theoretical models}

For such a broad range of applications of nuclear spallation, there is demand for the 
knowledge about the relevant quantities.
They are, e.g., the total and differential cross-sections for the production of various nuclides. Usually, it is difficult, time-consuming and/or costly to obtain experimentally desirable information. 
Therefore the existence of a reliable theory of the mechanisms contributing to the spallation reactions is necessary.

Nuclear spallation is a complicated phenomenon proceeding in the many-body, excited quantum system. Due to the lack of analytical mathematical formalism of such processes, the simplified models 
have to be used.
They contain contemporary knowledge about nuclear systems and nuclear reactions, 
usually with many unavoidable approximations. Models provide the solutions using the Monte Carlo simulations.

The models are developed and optimized in an iterative manner taking input from the information obtained experimentally.
Thus, it is crucial to provide valuable and precise experimental observables for models development and benchmarking.

% Thus, it is extremely important to provide the valuable and precise experimental observables  both for the models development and for their benchmarking.

\subsection{Types of models}

In the modeling of the proton ($p$) - target nucleus ($A$) collision most commonly the idea of Serber \cite{Serber} is adopted. It assumes
that the $p$ - $A$ collision proceeds in two steps of different time span:

\begin{itemize}
	\item The first and fast stage of the reaction consists of an intranuclear cascade (INC) of nucleon-nucleon and nucleon-pion collisions, leading to significant emission of nucleons, pions and nuclear clusters called Light Charged Particles (LCP). The excited residuum of the target nucleus appears in thermodynamical equilibrium.
	\item In the second stage (extended in time), various de-excitation processes of the residual excited nucleus are involved. The emission of single nucleons, LCP, heavier clusters (so-called Intermediate Mass Fragments - IMF) is possible due to evaporation, fission and/or fragmentation processes.
\end{itemize}
Fig \ref{first_step} presents the particle emission during the intranuclear cascade.

Various models used for simulation of both of these two stages of the reaction are available.

The dynamical - first step o reaction can be  simulated, e.g., with the following models: 
INCL++  \cite{INCLCugnon1981,INCLboudard2002intranuclear,INCLboudard2004new,INCLboudard2013new,INCLMancusi2014}, 
GiBUU \cite{GiBUUBuss2012}, 
UrQMD \cite{UrQMDBASS1998,UrQMDBleicher1999}, 
JAM \cite{JAM_NARA1999}, 
Bertini \cite{Bertini1963,Bertini1969}, 
CEM \cite{CEM_GUDIMA1983}, 
ISABEL \cite{Isabel_Yariv1979,Isabel_Yariv1981}. 

The slow deexcitation of the after-cascade 
remnant is calculated, e.g., in 
ABLA07 \cite{kelic2009abla07}, 
GEMINI++ \cite{CHARITY1988,Charity2010},
GEM2 \cite{FURIHATA2000,Furihata2002},
SMM \cite{SMMBondorf1995}.

The selection of the model depends on the 
projectile energy, target mass, resulting excitation energy, demanded quantity of interest but also on credibility of the model's results.



\label{first_step}
\begin{figure}
	\centering
	\begin{tikzpicture}[scale=0.65]
		\nucleus
		\proton{-8,0}
		\electron{1.2}{1.4}{260}
		\electron{4}{2}{30}
		\electron{5}{1}{60}
		\electron{5.5}{1.5}{150}
		\electron{4.8}{2.25}{80}
		\proton{7,5}
		\neutron{7,1.3}
		\nucleust
		\pion{7,3}
		\draw[color=Blue2](0,-5) node {Nucleus};
		\draw[blue](7.3,3.7) node {Pions};
		\draw[red](7.5,5.7) node {Proton};
		\draw[red](-8,1) node {Proton};
		\draw[green](7.5,2.0) node {Neutron};
		\draw[color=Purple1](7.5,-2.5) node {Gammas};
		\draw[color=Purple1](7.5,-3) node {$\gamma$};
		\draw (7.5,-0.3) node {Residua};
		\draw [->,color=gray,thick,dashed](-7.8,0) --(0,0); 
		\draw [->,color=red,dashed](0.5,0) --(6.8,4.8);
		\draw [->,color=blue,dashed] (0.5,-0.1) --(6.5,2.7) ;
		\draw [->,color=green,dashed] (0.5,-0.2) -- (6.5,1.2) ;
		\draw [->,,dashed] (0.5,-0.3) -- (6.5,-1.0) ;
		\draw [->,color=Purple1,decorate,decoration={snake,segment length=3mm, amplitude=1mm}] (0.5,-0.4) --(6.5,-2.9) ;
	\end{tikzpicture}
	\caption{Particle emission in the first stage of the spallation reaction}
\end{figure}


For example, the combination of the cascade model of INCL4.6 with the deexcitation model ABLA07
is able \cite{INCLboudard2013new} to describe with the precision of factor 2 the total cross-sections as well as the energy spectra and angular distributions of neutrons, protons and pions for the broad range of the bombarding energies of the light projectile 
(from 50 MeV to 5 GeV). The experimental yields of the remnant nuclei could be reproduced as well. 

However, these models - similarly to all other spallation models - meet a problem of the explanation and quantitative description of the non-equilibrium emission of complex light charged particles (LCP), \emph{i.e.},  $d$, $t$, $^3He$ and $^4He$ as well as of the intermediate mass fragments (IMF), \emph{i.e.}, particles heavier than $^{4}He$ but lighter than products of the fission.



\section{Questions that need to be answered in nuclear spallation research}

After many years of research in this field of experimental and theoretical groups, many fundamental questions are still unanswered. The main of them are:
%the few of them are:
\begin{itemize}
    \item Is the assumption about the two step 
    model justified or the scenario of the reaction is different (e.g., instantaneous formation of a few excited moving sources of particle emission \cite{budzanowski2010comparison}) ? 
	\item What are the mechanisms acting during the initial part of collision? Is the energy/momentum dissipation within the target nucleus realized only by the binary collisions or other mechanisms are present as well ? 
	\item How far it is justified to use the experimental nucleus-nucleus and nucleus-pion cross-sections measured in the vacuum for calculation of collision probability within the nuclear medium ?
	\item What are the mechanisms responsible 
	for the creation of fast composite nuclear particles ?
	\item How far the excited nuclear quantum system can be approximated by the statistical ensemble of point-like particles described by thermodynamics ? 
\end{itemize}	

%	Which of the models of the second stage of the reaction is the best in reproduction of selected observables (total and isotopic cross-sections) -
%	with the assumption that INCL is the best, realistic model of the first stage. On the basis of the literature data.
%	\item The extraction of new experimental DIFFERENTIAL cross-sections of the 
%	protons, pions and deuterons with the aim to check whether INCL is able to reproduce well high energy tail of the spectra (what is necessary to claim that INCL is a good model)
%	\item  To check whether other popular models of the 1st stage are equally good as the INCL model in description of these particles (NOTE: protons are the most abundant charged particles and heavier particles probably may be treated as not important in decision whether the main mechanism is reproduced)
%	\item To check how the properties of the residual nuclei after the 1st stage depend on the applied model 


%	\section{Motivation}
%	It is well known since a long time (cf. \emph{e.g.}, \cite{hyde1971characteristics,korteling1973energy}) that the spallation reactions with emission of LCP and IMF cannot be exclusively described by the emission from the equilibrated remnants of the first stage of the proton-nucleus collisions. Such an emission results in almost isotropic angular distributions whereas it was observed that the spectra of the composite ejectiles are built of two components: the low energy, quasi - isotropic component and the high energy, forward peaked one. Whereas the first component may be attributed to the emission of particles from the equilibrated heavy residuum of the intranuclear cascade the second component is interpreted as the contribution from a non-equilibrium emission of the products. It was found in the papers cited above as well as in many others,\emph{e.g.}\cite{bubak2007non,budzanowski2008competition,budzanowski2009variation,budzanowski2010comparison,fidelus2014sequential,fidelus2017non} that such character of spectra is observed for a broad range of targets (Al, Ni, Ag and Au) as well as for a broad range of the proton beam energies (1.2, 1.9 and 2.5 GeV) and concerns both, LCP and IMF. \par
%	A significant improvement has been achieved in the quantitative description of the non-equilibrium emission of LCP by the INCL4.3 model \cite{boudard2004new} due to the introduction of the surface coalescence of nucleons escaping from the intranuclear cascade stage. This success suggested that the same method might be, perhaps, extended to non-equilibrium emission of IMF. Therefore, further modifications of the INCL were undertaken \cite{boudard2013new}. The resulting version INCL++ of the INCL model considered emission of dynamically created clusters with the mass number A up to 12. However, it turned out that the maximal mass number A of considered IMF has to be in practice limited to values not larger than 8 due to a very long computing time. However, it turned out that very long computing time limited in practice the maximal mass number of considered IMF up to A not larger than 8. Moreover, the spectra predicted by the INCL++/INCL4.6 do not reproduce well the shape of the experimental spectra.  Their slope is always significantly smaller than that observed in the experiment \cite{sharma2016ranking} which leads to the overestimation of the cross-sections at high energies  of emitted IMF with simultaneous underestimation of them at small energies. It is therefore obvious that the invention of another microscopic model is desirable for realistic and efficient description of the non-equilibrium emission of intermediate mass fragments. To achieve such a goal it is necessary to collect %from experimental investigations as many as possible qualitative and quantitative experimental facts on the non-equilibrium processes underlying the IMF, LCP emission.

Some of these fundamental questions are addressed also in this thesis. The general assumption about the two step scenario of spallation process is adopted. 

\section{Organization of thesis}
The review of contemporary theoretical models used for description of the spallation reactions is presented in chapter \ref{chapter:2}. 
The following experimental and theoretical objectives of the present work are discussed in the further chapters of the thesis:

\begin{enumerate}[label=\roman*)]
	\item Studies of the first step of spallation reaction (chapters 3 and 4): 
	the experimental determination of the differential cross-section 
	($d^2\sigma/d\Omega dE$) 
	for emission of Hydrogen isotopes 
	as well as charged pions
	%$\pi^{+}$ and $\pi^{-}$ 
	from collisions of protons 
	with $Nb$ nuclei at $E_p$ = 3.5 GeV (experiment performed in the frame of HADES collaboration) and theoretical description of the obtained data;
%	For this topic 
    \item Second step of the reaction - phenomena governing the deexcitation of the remnant of the first step: 
    the theoretical analysis of the data published in the literature, which contain total isotopic cross-sections for production of nuclei from $Li$ to $Ba$ in $p$+$^{136}Xe$ collision at $E_p$ = 1 GeV. 
% ???    and differential as well as total cross-sections available for p+Ag collisions at Ep=0.48 GeV with production of isotopes of Li, Be, ... Mg nuclei. ???
    Results of this
	analysis were partially published in  \cite{singh2018predictive} and \cite{singh2019odd}. 
	For this subject the chapter 5 is devoted;
	\item Examination of the contribution of the non-equlibrium and equilibruim processes to the total production cross-section for various isotopes and its dependence on the ratio of protons and neutrons in the emitted object (chapter 6). For this aim the data of $p+Ag$ collisions at $E_p$ = 480 MeV published in \cite{Green1984} were utilized;
	\item Theoretical assessment of the experimental observation of variation  
	visible in the total production cross-section for various isotopes of similar mass number A known as Odd-Even Staggering of cross-section (OES). The experimental cross-section are adopted from \cite{Green1984}. This topic is presented in chapter 7.
\end{enumerate}


The thesis is concluded with the summary.

%was to study the mechanisms of pions, proton and nuclear composite particles production in proton-nucleus collisions. The studies were performed partially experimental – utilizing the data of the HADES collaboration and analysing that data to extract the cross-section for p,d,t, and pions, and to some extent theoretical, in the sense that the validation of the existing theoretical models is performed. 
%Two, 
%qualitatively 
%different investigations were undertaken in the present thesis:


